\documentclass{article}
\title{Rolling dice}
\author{Fabio Massimo Ercoli
	\footnote{
		More on me:
		\href{https://github.com/fax4ever}{my GitHub page} -
		\href{https://www.linkedin.com/in/fabioercoli}{my Linkedin page}.
}}
\date{December 30\textsuperscript{th} 2024}
\usepackage{listings}
\usepackage{hyperref}
\usepackage{color, colortbl}
\usepackage{graphicx}
\graphicspath{ {./images/} }
\usepackage{subfig}

\begin{document}
\maketitle
\thispagestyle{empty}

\section{Protocol basic ideas}

Most of the time, in security scenarios Alice and Bob collaborate
implementing security strategies against possible attackers (such as Eve).
This case is quite different since Alice and Bob are competing each other,
and they can try to trick each other.

\paragraph{Non repudiation}

In my opinion a crucial security property to implement the protocol is 
the \emph{non repudiation}. We want to be sure that once a dice is cast
by a given player (e.g., Alice and Bob) the outcome must not be disputable.
In particular, we want that all the entities knowing the public key of the caster
are able to check if a given action was really made by the caster,
with no possibility of making mistakes, with the only exception,
not considered in this protocol, that the private key of the caster
has been compromised.

\paragraph{The (non)randomness of the casts}

Cast outcomes should be produced in such a way 
their values are taken uniformly at random in the 1 to 6 interval.
There are probably several ideas that can be used to 
try to enforce it.
My idea here is to produce two numbers for each cast
of Alice and Bob, one from Alice and one from Bob, and then
xor-ing them.

\begin{itemize}
	\item If both Alice and Bob produce both random numbers, of course
	the xor of two random numbers is a random number, so it is fine.
	\item If one of them produces a random number and the other tries
	to trick producing some number that is not random, and the one
	that produces the non-random value does not know the value
	produced by the other before casting the dice, again we're in
	a good shape, since the result will be a random value again.
	\item If both try to trick producing some non-random values,
	if none of them know the non-random value produced 
	by the other before to cast her/his choice, even if the result
	is not random, I think that it is impossible to get some
	advantages for the trickier to have more chances to win.
\end{itemize}

\paragraph{Synchronization}
	
According to what we said in the last paragraph we need to synchronize
the casts so that none of the players can see the cast of the other player
before to cast her/his dice.
The solution we adopt here is to use a third party process to collect
both the contribution related to a single pair of casts, and only
when it receives the the contributions for both the players,
it can share them with the same players.

About the security for this third actor, we can call it Carol, we assume that
it is safe, maybe it is a bastion host. In any case I think that the protocol
can be extended adding for instance the certificate of this third
party server, but I don't do that here for sake of simplicity.

\section{Proposed protocol}

\subsection{Assumptions, hipothesys, limitations}

First of all we assume that:
\begin{itemize}
	\item The Alice node, the Bob node, the Carol node correctly have the public key of Alice and Bob.
	Imaging that they have safely obtained them using PKI.
	\item The Carol node is trusted.
	\item We don't expect that the rolls of the dices is uniformly at random in all cases.
	We expect it in most of the cases and then this does not happen, we expect that
	the player that produced the non-random value didn't get any advantage from it.
\end{itemize}	

\subsection{Protocol Data structure}

We need to imagine a data structure of an Object Oriented class to
denote (and store all the state of) a \emph{Game}.

\noindent The \emph{Game} entity contains:
\begin{itemize}
	\item \textbf{id}: An identifier of the game, maybe an UUID. This value is used to identify a game instance unequivocally.
	\item \textbf{aliceRolls}: A list (ordered) of what we can call \emph{RollDice} items for Alice
	\item \textbf{bobRolls}: A list (ordered) of what we can call \emph{RollDice} items for Bob
\end{itemize}

\noindent Where the size of the lists is equal to the size of k (the number of dice rolled).

\noindent The \emph{RollDice} entity contains:
\begin{itemize}
	\item \textbf{roll}: A random generated value that denotes the player roll.
	\item \textbf{antiRoll}: A random generated value that will be xor-ed with the other player's roll in the same position
	of the \emph{RollDice} item lists.
\end{itemize}

\subsection{Compute the rolling dice outcomes and the winner}

The roll dice outcomes and then the winner can be computed only if we
have both the roll dice items from Alice and Bob.

The length of the two lists \emph{aliceRolls} and \emph{bobRolls} must be the same (k).
For each i position from the first to the last we compute the values xor-ing the roll values
with the corresponding \emph{antiRoll} of the other player, then we compute the module 6
and we add 1 to get an outcome in the interval [1 to 6]:

\begin{itemize}
	\item the i-th Alice roll outcome = ( \emph{aliceRolls}[i].roll $\oplus$ \emph{bobRolls}[i].antiRoll \% 6 ) + 1
	\item the i-th Bob roll outcome = ( \emph{bobRolls}[i].roll $\oplus$ \emph{aliceRolls}[i].antiRoll \% 6 ) + 1
\end{itemize}	

The computation of the winner is trivially done summing the roll outcomes of both players and
selecting the player who has the higher sum.

\subsection{Protocol dynamic}

We can describe the dynamic of the protocol with the following steps:

\begin{itemize}
	\item When the game starts, Alice and Bob need to agree on the same game id.
	\item Alice computes the \emph{aliceRolls},
		\subitem she signs using her private key the \emph{aliceRolls} + the game id
		\subitem she sends the list tougher with the game id and the sign to Carol
	\item Concurrently, Bob computes the \emph{bobRolls},
		\subitem he signs using his private key the \emph{bobRolls} + the game id
		\subitem he sends the list tougher with the game id and the sign to Carol
	\item When Carol delivers a pair of  \emph{aliceRolls} \emph{bobRolls} lists,
		for which she has verified the sign:
		\subitem She sends \emph{bobRolls} tougher with the game id and the Bob's signature to Alice
		\subitem  Concurrently, she sends \emph{aliceRolls} tougher with the game id and the Alice's signature to Bob
	\item When Alice delivers the \emph{bobRolls} is able to verify the authenticity and compute the rolling dice outcomes and the winner and the same symmetrically is true for Bob.
\end{itemize}

At the end of the process neither Alice nor Bob can repudiate their rolls, this guarantees 
the consensus on the rolling dice outcomes and the winner for the given game.

\section{Extra: Third party role}

In this implementation of the protocol the role of Carol is very limited,
she basically only implement an exchange pattern for the messages containing the lists 
provided by Alice and Bob.

We could have provided more responsibilities to this component,
for instance we could have generated the random numbers from Carol node.
I didn't do that because usually limiting the load on third party nodes is
a good practice to improve the scalability of the solution,
since in this way we spread more the computation costs among different nodes.
\end{document}		