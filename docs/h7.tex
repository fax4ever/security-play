\documentclass{article}
\title{Testing a TLS connection}
\author{Fabio Massimo Ercoli
	\footnote{
		More on me:
		\href{https://github.com/fax4ever}{my GitHub page} -
		\href{https://www.linkedin.com/in/fabioercoli}{my Linkedin page}.
}}
\date{December 25\textsuperscript{th} 2024}
\usepackage{listings}
\usepackage{hyperref}
\usepackage{color, colortbl}

\begin{document}
\maketitle
\thispagestyle{empty}
	
\section{TLS handshake runs}

I created the script file \emph{tls-checker.sh} to test the TLS handshakes
on 20 different sites:

\begin{verbatim}
google.com, facebook.com, redhat.com, infinispan.org, almaviva.it, 
www.eng.it, edenred.it, esa.int, theguardian.com, repubblica.it, 
huffingtonpost.it, quotidianolavoce.it, baraondanews.it, 
mit.edu, stanford.edu, uniroma1.it, unicusano.it, 
chiaveorgonica.it, diocesiportosantarufina.it, hwpviewer.hbedu.co.kr
\end{verbatim}

\noindent The core of the script is the \emph{checkTLS} function:

\begin{lstlisting}[language=bash]
#!/usr/bin/env bash
set -e
openssl version

function checkTLS() {
	start=`date +%s%N`
	echo | openssl s_client -connect $1:443 2>/dev/null
	end=`date +%s%N`
	runtime=$((end-start))
	echo "$1 => $runtime"
}

checkTLS google.com
checkTLS facebook.com
...
\end{lstlisting}

\noindent I collected the output in a text file \emph{tls-checkers.txt}
\begin{lstlisting}[language=bash]
./tls-checker.sh > tls-checkers.txt
\end{lstlisting}

\section{TLS handshake times}

From the previous result it is possible to extract the times, 
they are collected as microseconds (1 microsecond = $10^{-9}$ seconds).

\begin{lstlisting}[language=bash]
cat tls-checkers.txt | grep "=>"
cat tls-checkers.txt | grep "New, "
\end{lstlisting}

\noindent Here is my result:

\bigbreak
\begin{tabular}{ c c c c }
\textbf{Connection} & \textbf{Time} & \textbf{TLSv} & \textbf{Cipher}  \\ 
\hline
google.com & 115.264.962 & TLSv1.3 & TLS AES 256 GCM SHA384 \\
facebook.com & 066.053.836 & TLSv1.3 & TLS CHACHA20 POLY1305 SHA256 \\
redhat.com & 568.495.098 & TLSv1.3 & TLS AES 256 GCM SHA384 \\
infinispan.org & 097.765.850 & TLSv1.3 & TLS AES 128 GCM SHA256 \\
almaviva.it & 216.864.263 & TLSv1.2 & ECDHE-RSA-AES128-GCM SHA256 \\
www.eng.it & 109.731.688 & TLSv1.3 & TLS AES 128 GCM SHA256 \\
edenred.it & 108.295.706 & TLSv1.3 & TLS AES 256 GCM SHA384 \\
esa.int & 204.531.698 & TLSv1.2 & ECDHE-RSA-AES256-GCM SHA384 \\
theguardian.com & 123.894.304 & TLSv1.3 & TLS AES 128 GCM SHA256 \\
repubblica.it& 86.482.951 & TLSv1.3 & TLS AES 128 GCM SHA256 \\
huffingtonpost.it & 269.442.781 & TLSv1.2 & ECDHE-ECDSA-AES128-GCM SHA256 \\
quotidianolavoce.it & 661.609.322 & TLSv1.3 & TLS AES 256 GCM SHA384 \\
baraondanews.it & 170.123.645 & TLSv1.3 & TLS AES 256 GCM SHA384 \\
mit.edu & 1040.751.680 & TLSv1.2 & ECDHE-RSA-AES256-GCM SHA384 \\
stanford.edu & 912.750.222 & TLSv1.2 & AES128-GCM SHA256 \\
uniroma1.it & 137.912.714 & TLSv1.2 & ECDHE-RSA-AES256-GCM SHA384 \\
unicusano.it & 081.213.441 & TLSv1.3 & TLS AES 128 GCM SHA256 \\
chiaveorgonica.it & 162.517.465 & TLSv1.3 & TLS AES 256 GCM SHA384 \\
diocesiportosantarufina.it & 141.360.897 & TLSv1.3 & TLS AES 256 GCM SHA384 \\
hwpviewer.hbedu.co.kr & 1636.988.593 & TLSv1.3 & TLS AES 256 GCM SHA384 \\
\end{tabular}
\bigbreak

\section{Describes two connections in detail}

We analyze in more detail two connections: \emph{facebook.com} and \emph{redhat.com}.
To have more information about the handshake process, we run the \emph{openssl c\_client}
command passing the options: \emph{-trace -debug -state}.
Moreover, we add the options \emph{-tls1\_3}
to be sure of using TLS 1.3 (as requested), even if the higher protocol should be
applied by default in the version negotiation phase.
I collected the output in a text file \emph{tls-checkers-deep.txt},
after than I split it into \emph{facebook.txt} and \emph{redhat.txt}.

\subsection{ClientHello $\rightarrow$}

At the beginning the client sends a \emph{ClientHello} message.
There is a lot of information here:

 \begin{itemize}
 	\item The protocol version \emph{TLS 1.2}, with extension indicating \emph{TLS 1.3}
	\item A random value (28 bits), plus a timestamp \emph{gmt\_unix\_time} (4 bits)
	\item A 32-bits session ID
	\item The supported cipher suites by the client (in order of preference)
	\item In the extensions we can find:
		\subitem The name of the server
		\subitem Supported version (in this case only 1.3 - since we used this option!)
		\subitem Supported groups
		\subitem Supported signature algorithms
\end{itemize}

\subsection{ServerHello $\leftarrow$}

From the server we receive a \emph{ServerHello} message.
In its payload we can find:

 \begin{itemize}
 	\item The protocol version \emph{TLS 1.2}, with extension indicating \emph{TLS 1.3}
	\item A new random value (28 bits), plus a timestamp \emph{gmt\_unix\_time} (4 bits)
	\item A 32-bits session ID (the same of the client)
	\item The chosen cipher suite: 
		\subitem \emph{TLS\_AES\_256\_GCM\_SHA384} for Red Hat
		\subitem \emph{TLS\_CHACHA20\_POLY1305\_SHA256} for Facebook
	\item In the extensions we can find:
		\subitem Chosen TLS version:  1.3
		\subitem Chosen group: \emph{ecdh\_x25519}
		\subitem The server ephemeral public key for the key exchange
\end{itemize}

\subsection{Other server messages $\leftarrow$}

After that the server sends the \emph{EncryptedExtensions} and
the \emph{Certificate} messages.
We can see here the certificates containing the servers' public keys.

For Red Hat, having x501 name:
\begin{verbatim}
	C = US, ST = North Carolina, L = Raleigh, O = "Red Hat, Inc.", CN = redhat.com
\end{verbatim}

It is signed by:

\begin{verbatim}
C = US, O = DigiCert Inc, CN = DigiCert Global G2 TLS RSA SHA256 2020 CA1
\end{verbatim}

For Facebook we have \emph{CompressedCertificate}
in place of \emph{Certificate}, this probably allows it to speed up the handshake.
The subject is:
\begin{verbatim}
subject=C=US, ST=California, L=Menlo Park, O=Meta Platforms, Inc., CN=*.facebook.com
\end{verbatim}

and the issuer is:

\begin{verbatim}
C=US, O=DigiCert Inc, OU=www.digicert.com, CN=DigiCert SHA2 High Assurance Server CA
\end{verbatim}

The \emph{CertificateVerify} contains the chain of certificates to verify the certificate.

\subsection{Final handshake}

Both parties sends each other a \emph{Finished} message,
containing the \emph{verify\_data} field used to verify and complete
the handshake.

The result of the handshake for RedHat is:

\begin{verbatim}
No client certificate CA names sent
Peer signing digest: SHA256
Peer signature type: RSA-PSS
Server Temp Key: X25519, 253 bits
---
SSL handshake has read 4052 bytes and written 331 bytes
Verification: OK
---
New, TLSv1.3, Cipher is TLS_AES_256_GCM_SHA384
Server public key is 4096 bit
This TLS version forbids renegotiation.
Compression: NONE
Expansion: NONE
No ALPN negotiated
Early data was not sent
Verify return code: 0 (ok)
\end{verbatim}

The result of the handshake for Facebook is:

\begin{verbatim}
No client certificate CA names sent
Peer signing digest: SHA256
Peer signature type: ECDSA
Server Temp Key: X25519, 253 bits
---
SSL handshake has read 2449 bytes and written 317 bytes
Verification: OK
---
New, TLSv1.3, Cipher is TLS_CHACHA20_POLY1305_SHA256
Server public key is 256 bit
This TLS version forbids renegotiation.
Compression: NONE
Expansion: NONE
No ALPN negotiated
Early data was not sent
Verify return code: 0 (ok)
\end{verbatim}

We can noticed that the Facebook's server public key is much shorter,
in general less data is needed and the handshake is faster.

\section{Classify the results}

I used \href{https://www.ssllabs.com/ssltest/}{SSL labs} to collect security
results on the connections. The result can be clustered
according to colors: green(A/A+), cyan(B - key ex), yellow(B - protocol), red(F), magenta(T):

\bigbreak
\begin{tabular}{ c c c c c c } 
\textbf{Connection} & \textbf{Certificate} & \textbf{Protocol} & \textbf{KeyEx} & \textbf{Ciphers} & \textbf{Overall}  \\ 
\hline
\rowcolor{yellow}
google.com & 100 & 70 & 90 & 90 & \textbf{B} \\
\rowcolor{yellow}
facebook.com & 100 & 70 & 90 & 90 & \textbf{B} \\
\rowcolor{green}
redhat.com & 100 & 100 & 90 & 90 & \textbf{A+} \\
\rowcolor{green}
infinispan.org & 100 & 100 & 90 & 90 & \textbf{A} \\
\rowcolor{magenta}
almaviva.it & 0 & 70 & 90 & 90 & \textbf{T} \\
\rowcolor{green}
www.eng.it & 100 & 100 & 90 & 90 &  \textbf{A+} \\
\rowcolor{green}
edenred.it & 100 & 100 & 90 & 90 &  \textbf{A} \\
\rowcolor{green}
esa.int & 100 & 100 & 90 & 90 &  \textbf{A} \\
\rowcolor{green}
theguardian.com & 100 & 100 & 90 & 90 &  \textbf{A+} \\
\rowcolor{green}
repubblica.it & 100 & 100 & 90 & 90 &  \textbf{A} \\
\rowcolor{yellow}
huffingtonpost.it & 100 & 70 & 90 & 90 &  \textbf{B} \\
\rowcolor{green}
quotidianolavoce.it & 100 & 100 & 90 & 90 &  \textbf{A} \\
\rowcolor{yellow}
baraondanews.it & 100 & 70 & 90 & 90 &  \textbf{B} \\ 
\rowcolor{green}
mit.edu & 100 & 100 & 90 & 90 &  \textbf{A} \\ 
\rowcolor{cyan}
stanford.edu & 100 & 100 & 70 & 90 &  \textbf{B} \\ 
\rowcolor{green}
uniroma1.it & 100 & 100 & 90 & 90 &  \textbf{A} \\ 
\rowcolor{yellow}
unicusano.it & 100 & 70 & 90 & 90 &  \textbf{B} \\ 
\rowcolor{yellow}
chiaveorgonica.it & 100 & 70 & 90 & 90 &  \textbf{B} \\
\rowcolor{magenta}
diocesiportosantarufina.it & 0 & 100 & 90 & 90 &  \textbf{T} \\
\rowcolor{red}
hwpviewer.hbedu.co.kr & 100 & 70 & 0 & 0 &  \textbf{F} \\
\end{tabular}
\bigbreak

\subsection{A/A+ green connections}

Those connections are good in every aspect. In particular the A+ ones have also a \emph{DNS Certification Authority Authorization (CAA) Policy} and \emph{HTTP Strict Transport Security (HSTS)} defined.
I collected the output in a text file \emph{tls-checkers-deep.txt}

\subsection{B yellow connections}

Those connections are good on certificate, key exchange and ciphers, while  
the protocol can be improved on the protocol side.
The reason is that those connections support also TLS 1.1 and 1.0
that are now considered weak.
According to the tool, even if those versions can be used, the
downgrade attack prevention in place.
The reason probably is not a lack of skills of the provider of those
services, but the strategic choice of support a wider range
of clients.

\subsection{B cyan connection}

In this case certificate, protocol and ciphers are good, while the key exchange process
of the TLS handshake can be improved since the server supports the Forward Secrecy
only for a few browsers.

\subsection{F red connection}

This is the case of the connection \emph{hwpviewer.hbedu.co.kr},
as the name suggested it is probably not a production portal.
It is nice to analyze it, since we can found very deep vulnerabilities
on the key exchange process and on ciphers used.

\subsection{T magenta connections}

The problem of those connections is the fact that
the server certificates have been expired.
This is not always true in all cases.
From instance in the case of Almamiva portal
the certificate is no longer valid only if the
connection is served by the \emph{*.awsglobalaccelerator.com}
host. The \emph{Valid until} certificate field contains the value:

\begin{verbatim}
Fri, 15 Nov 2024 23:59:59 UTC (expired 1 month and 8 days ago)   
EXPIRED
\end{verbatim}

\noindent In this case the score for the certificate is of course 0.

\end{document}		